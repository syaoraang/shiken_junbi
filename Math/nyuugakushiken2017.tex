\documentclass[11pt, letterpaper]{article}
\title{\textbf{Tokio university's entrance exam 2017}}
\author{Acosta Domínguez Jorge}
\date{}

\usepackage{amsmath}
\usepackage{tabto}
\usepackage{tasks}
\usepackage{geometry} %this reduces the amount of margin
\newcommand{\itab}[0]{\hspace{2em}}

\begin{document}
\pagenumbering{gobble}
\maketitle
\newpage
\pagenumbering{arabic}

\section{Mathmatics exam}
\subsection{Lineal algebra}

Suppose that three-dimensional vectors $\left(
\begin{matrix}
x_n\\y_n\\z_n\\
\end{matrix}\\
\right)$
satisfy the equation
\begin{equation*}
\left(
\begin{matrix}
x_{n+1}\\y_{n+1}\\z_{n+1}
\end{matrix}
\right)
=A\left(
\begin{matrix}
x_n\\y_n\\z_n\\
\end{matrix}\\
\right)\itab n = 0, 1, 2, ...,
\end{equation*}
where $x_0$, $y_0$, $z_0$ and $\alpha$ are real numbers, and
\begin{equation*}
A = \left(\begin{matrix}
1-2\alpha & \alpha & \alpha\\
\alpha & 1-\alpha & 0\\
\alpha & 0 & 1-\alpha
\end{matrix}\right),\itab0<\alpha<\frac{1}{3}.
\end{equation*}
Answer the following questions.
\settasks{
	counter-format=(tsk),
	label-width=3ex
}
\begin{tasks}
\task{Express $x_n + y_n + z_n$ using $x_0, y_0$ and $z_0$}
\task{Obtain the eigenvalues $\lambda_1, \lambda_2$ and $\lambda_3$, and their corresponding eigenvectors $v_1, v_2, v_3$ of the matrix $A$}
\task{Express the matrix $A$ using $\lambda_1, \lambda_2, \lambda_3, v_1, v_2, v_3$.}
\task{Express $\left(\begin{matrix}
x_n\\y_n\\z_n
\end{matrix}\right)$}
using $x_0, y_0, z_0$ and $\alpha$.
\task{Obtain $\lim_{n\to\infty}
\left(\begin{matrix}
x_n\\y_n\\z_n
\end{matrix}\right)$}.
\task{Regard
\begin{equation*}
f(x_0, y_0, z_0) = \frac{(x_n, y_n, z_n)\left(\begin{matrix}x_{n+1}\\y_{n+1}\\z_{n+1}\end{matrix}\right)}{(x_n, y_n, z_n)\left(\begin{matrix}x_n\\y_n\\z_n\end{matrix}\right)}
\end{equation*}
as a function of $x_0, y_0$ and $z_0$. Obtain the maximum and the minimum values of $f(x_0, y_0, z_0)$, where we assume that $x_0^2 + y_0^2 + z_0^2 \neq 0$.}
\end{tasks}\newpage

\subsubsection{Resolution}
\begin{tasks}
\task{
Using $n=0$
\begin{align*}
x_1 &= (1-2\alpha)x_0+\alpha y_0 + \alpha z_0\\
y_1 &= \alpha x_0 + (1-\alpha)y_0\\
z_1 &= \alpha x_0 + (1-\alpha)z_0
\end{align*}
adding both sides of the equations:
\begin{align*}
x_1 + y_1 + z_1 &= (1-2\alpha)x_0 + \alpha x_0 + \alpha x_0 + \alpha y_0 + (1-\alpha) y_0 + \alpha z_0 + (1-\alpha)z_0\\
x_1 + y_1 + z_1 &= x_0(\alpha + \alpha -2\alpha + 1) + y_0(\alpha + 1 -\alpha) + y_0(\alpha + 1 -\alpha)\\
x_1 + y_1 + z_1 &=x_0 + y_0 + z_0
\end{align*}
due that with $n=1$ we get $x_1 + y_1 + z_1 =x_0 + y_0 + z_0$ we can say that:
$$x_n + y_n + z_n=x_0 + y_0 + z_0$$
}
\task{
For this part we have to recall that $|A-I\lambda|=0$. So, we substract $I\lambda$ from our A to get this:
\begin{equation*}
\left|\begin{matrix}
1-2\alpha-\lambda & \alpha & \alpha\\
\alpha & 1-\alpha-\lambda & 0\\
\alpha & 0 & 1-\alpha-\lambda
\end{matrix}\right| = 0
\end{equation*}
By getting the determinant of $|A-I\lambda|$ we get $(\lambda+1)(\lambda-1+3\alpha)(\lambda-1+\alpha)=0$
So the values of $\lambda$ are:
\begin{itemize}
\item{$\lambda_3 = 1-\alpha$}
\item{$\lambda_2 = 1-3\alpha$}
\item{$\lambda_1 = 1$}
\end{itemize}
Now, substituting each of the $\lambda$ values in $A-I\lambda$ and resolving the equations system we get the following vectors:\\
For $\lambda = 1-\alpha$
\begin{equation*}
\left(\begin{matrix}
1-2\alpha-(1-\alpha) & \alpha & \alpha\\
\alpha & 1-\alpha-(1-\alpha) & 0\\
\alpha & 0 & 1-\alpha-(1-\alpha)
\end{matrix}\right)
,\itab v_1 = \left(\begin{matrix}
0\\-1\\1
\end{matrix}\right)
\end{equation*}
For $\lambda = 1-3\alpha$
\begin{equation*}
\left(\begin{matrix}
1-2\alpha-(1-3\alpha) & \alpha & \alpha\\
\alpha & 1-\alpha-(1-3\alpha) & 0\\
\alpha & 0 & 1-\alpha-(1-3\alpha)
\end{matrix}\right)
,\itab v_2 = \left(\begin{matrix}
-2\\1\\1
\end{matrix}\right)
\end{equation*}
For $\lambda = 1$
\begin{equation*}
\left(\begin{matrix}
1-2\alpha-1 & \alpha & \alpha\\
\alpha & 1-\alpha-1 & 0\\
\alpha & 0 & 1-\alpha-1
\end{matrix}\right)
,\itab \itab \itab \itab \itab v_3 = \left(\begin{matrix}
1\\1\\1
\end{matrix}\right)
\end{equation*}
}
\task{
Since we can decompose any matrix $A$ in $A=Q\Lambda Q^{-1}$ where $\Lambda$ is the diagonal matrix made of the eigenvalues that we found, and $Q$ is the the matrix made of eigenvectors that we already found. We simply have to build the Q matrix and get its inverse.\\
We have that $Q=\left(\begin{matrix}
0 & -2 & 1\\
-1 & 1 & 1\\
1 & 1 & 1
\end{matrix}\right)$, with this we only have to find $Q^{-1}$.\\
Altought there are differents ways to get the inverse of a matrix the used for this exercise is the gauss-Jordan one. Solving we have:\\
$$\frac{1}{6}\left(\begin{matrix}
0 & -3 & 3\\
-2 & 1 & 1\\
2 & 2 & 2
\end{matrix}\right)$$
So this way we could represent $A$ using its eigenvalues and its eigenvectors\\
\begin{equation*}
A = 
\frac{1}{6}
\left(\begin{matrix}
0 & -2 & 1\\
-1 & 1 & 1\\
1 & 1 & 1
\end{matrix}\right)
\left(\begin{matrix}
1-\alpha & 0 & 0\\
0 & 1-3\alpha & 0\\
0 & 0 & 1
\end{matrix}\right)
\left(\begin{matrix}
0 & -3 & 3\\
-2 & 1 & 1\\
2 & 2 & 2
\end{matrix}\right)
\end{equation*}
}
\task{
For fast analisys lets rename $\left(\begin{matrix}
x_n\\y_n\\z_n
\end{matrix}\right)$ as $V_n$.\\
This way we have that \begin{equation}
V_{n+1} = AV_n
\end{equation}
If we set $n=0$ we have $V_1=AV_0$. With this we can deduce that \begin{equation}
V_n = A^nV_0
\end{equation}\\
If we apply it to $v_{n+1}$ we have that $V_{n+1} = AV_n=A(A^nV_0)=A^{n+1}V_0$. This way we can prove the validity of our deduction.
So, to express $x_n + y_n + z_n$ using $x_0, y_0$ and $z_0$ we first need to find the value of $A^n$\\
Fortunately we know that \begin{equation}
A^n = Q\Lambda^nQ^{-1}
\end{equation}.
$$
A^n = 
\frac{1}{6}
\left(\begin{matrix}
0 & -2 & 1\\
-1 & 1 & 1\\
1 & 1 & 1
\end{matrix}\right)
\left(\begin{matrix}
1-\alpha & 0 & 0\\
0 & 1-3\alpha & 0\\
0 & 0 & 1
\end{matrix}\right)^n
\left(\begin{matrix}
0 & -3 & 3\\
-2 & 1 & 1\\
2 & 2 & 2
\end{matrix}\right)$$
$$
A^n = 
\frac{1}{6}
\left(\begin{matrix}
0 & -2 & 1\\
-1 & 1 & 1\\
1 & 1 & 1
\end{matrix}\right)
\left(\begin{matrix}
(1-\alpha)^n & 0 & 0\\
0 & (1-3\alpha)^n & 0\\
0 & 0 & 1
\end{matrix}\right)
\left(\begin{matrix}
0 & -3 & 3\\
-2 & 1 & 1\\
2 & 2 & 2
\end{matrix}\right)
$$
$$
A^n = \frac{1}{6}
\left(\begin{matrix}
4(1-3\alpha)^n+2 & -2(1-3\alpha)^n+2 & -2(1-3\alpha)^n+2\\
-2(1-3\alpha)^n+2 & 3(1-\alpha)^n+(1-3\alpha)^n+2 & -3(1-\alpha)^n+(1-3\alpha)^n+2\\
-2(1-3\alpha)^n+2 & -3(1-\alpha)^n+(1-3\alpha)^n+2 & 3(1-\alpha)^n+(1-3\alpha)^n+2
\end{matrix}\right)
$$
So, if we replace this in $3$ we have:
$$
V_n = \frac{1}{6}
\left(\begin{matrix}
4(1-3\alpha)^n+2 & -2(1-3\alpha)^n+2 & -2(1-3\alpha)^n+2\\
-2(1-3\alpha)^n+2 & 3(1-\alpha)^n+(1-3\alpha)^n+2 & -3(1-\alpha)^n+(1-3\alpha)^n+2\\
-2(1-3\alpha)^n+2 & -3(1-\alpha)^n+(1-3\alpha)^n+2 & 3(1-\alpha)^n+(1-3\alpha)^n+2
\end{matrix}\right)V_0
$$
Substituying $V_n$ to its original form and multiplying we get the following:
\begin{displaymath}
\left(\begin{matrix}x_n\\y_n\\z_n\end{matrix}\right)=
\frac{1}{6} \left(\begin{matrix}
4(1-3\alpha)^n+2 & -2(1-3\alpha)^n+2 & -2(1-3\alpha)^n+2\\
-2(1-3\alpha)^n+2 & 3(1-\alpha)^n+(1-3\alpha)^n+2 & -3(1-\alpha)^n+(1-3\alpha)^n+2\\
-2(1-3\alpha)^n+2 & -3(1-\alpha)^n+(1-3\alpha)^n+2 & 3(1-\alpha)^n+(1-3\alpha)^n+2
\end{matrix}\right)\left(\begin{matrix}x_0\\y_0\\z_0\end{matrix}\right)
\end{displaymath}
\begin{align*}
x_n &= \frac{1}{3}((x_0+y_0+z_0)+(1-3\alpha)^n(x_0-y_0-z_0))\\
y_n &= \frac{1}{3}(x_0+y_0+z_0)+\frac{1}{6}(1-3\alpha)^n(-2x_0+y_0+z_0)+\frac{1}{2}(1-\alpha)^n(y_0-z_0)\\
z_n &= \frac{1}{3}(x_0+y_0+z_0)+\frac{1}{6}(1-3\alpha)^n(-2x_0+y_0+z_0)+\frac{1}{2}(1-\alpha)^n(-y_0+z_0)
\end{align*}
}
\end{tasks}
\end{document}
